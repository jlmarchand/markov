\documentclass[10pt,fleqn, svgnames]{article}
\usepackage{amsfonts,amssymb}
%\usepackage{euscript}
\usepackage[utf8]{inputenc}
\usepackage[T1]{fontenc} 
\usepackage{lmodern}
\usepackage[frenchb]{babel}
\usepackage{xspace}
\usepackage{mathtools}
\usepackage{fullpage}
\usepackage{comment}
\usepackage{graphicx}
\usepackage[body={160mm,230mm}, a4paper,centering]{geometry}
\usepackage{amsthm}
\usepackage{tikz}
\usepackage{pgfplots,filecontents}
\usetikzlibrary{shapes,snakes}
\usetikzlibrary{trees}
\usetikzlibrary{decorations.pathreplacing}
\usetikzlibrary{calc,positioning,backgrounds}
\usepgfplotslibrary{colormaps}
\usepackage{xcolor}

\usepackage{multicol}
\usepackage{multirow}
\usepackage{array}
\usepackage{xcolor,colortbl}
\usepackage{marvosym}

\newcommand{\re}{\mathbb{R}}
\newcommand{\z}{\mathbb{Z}}
\newcommand{\n}{\mathbb{N}}
\newcommand{\q}{\mathbb{Q}}

\theoremstyle{definition}
\newtheorem{exo}{Exercice}

 

\setlength{\hoffset}{-25pt}         
\setlength{\voffset}{-40pt}         

\begin{document}
\begin{exo}
On s’int\'eresse \`a une forêt compos\'ee de deux esp\`eces d’arbres, $E_1$ et $E_2$. Lorsqu’un arbre meurt, un nouveau grandit \`a
sa place mais il peut être de l’une ou l’autre des deux esp\`eces. Ceux de la premi\`ere esp\`ece
ayant une longue dur\'ee de vie, on suppose que 1\% d’entre eux meurt chaque ann\'ee alors
que ce taux est de 5\% pour la deuxi\`eme esp\`ece. Mais ces derniers grandissant plus rapidement r\'eussiront plus souvent \`a occuper une place laiss\'ee vacante : on suppose que 75\%
des places vacantes sont prises par un arbre de la deuxi\`eme esp\`ece contre seulement 25\%
pour un arbre de la premi\`ere esp\`ece.
\begin{enumerate}
\item Expliquer comment l'occupation d'un emplacement de cette foret peut être modélisé par une chaîne
de Markov $(X_t)_{t\geq 0}$ \`a deux \'etats $E_1$ et $E_2$ et justifier la formule suivante :
$$\mathbb P(X_{t+1} = E_1|X_t = E_1)=0, 99 + 0, 01 \times 0, 25 = 0, 9925.$$
\item En d\'eduire la matrice de transition P de la chaîne de Markov .
\item Tracer le graphe correspondant.
%~ \item Supposons que la probabilité d'avoir au début de l'étude un arbre d'espèce $E_1$ soit $\mu_1 = .1$, pour chaque espèce, déterminer les probabilités d'avoir un arbre  de cette espèce au bout de trois ans.
\item $\pi_1 = \begin{pmatrix}0, 625 &0, 375\end{pmatrix}$ est-elle une distribution stationnaire pour cette chaîne de Markov ?
\item $\pi_2 = \begin{pmatrix}0, 01 &0, 99\end{pmatrix}$ est-elle une distribution stationnaire pour cette chaîne de Markov ?
\item Que peut-on en déduire quand au comportement de la forêt en temps long ?
\end{enumerate}
\end{exo}

\begin{exo}
	On considère une chaîne de Markov $(Xn)_{n\geq0}$ sur $\{1, \dots, 7\}$ de matrice de transition $Q$
donnée par
$$Q =
\begin{pmatrix}
\frac12 &\frac14 &0 &\frac14 &0 &0 &0 \\
\frac12 &0 &0 &0 &0 &0 &\frac12\\
0& 0 &\frac18 &0 &\frac78 &0 &0\\
\frac14 &0 &0 &0 &0 &0 &\frac34\\
0 &\frac19 &\frac79 &0 &0 &\frac19 &0\\
0 &0 &0 &0 &0 &1 &0\\
0 &0 &0 &1 &0 &0 &0\end{pmatrix}$$
\begin{enumerate}
\item Dessiner le graphe de la chaîne de Markov associée en précisant les probabilités de transitions
entre les différents états.
\item Déterminer les classes d’états récurrents et transitoires.
\item La chaîne est-elle irréductible ?
\item Calculer $\mathbb P_3(X_2 = 6)$ et $\mathbb P_1(X_2 = 7)$ (où $\mathbb P_i$ désigne la probabilité sachant que $X_0=i$).
\end{enumerate}
\end{exo}


\end{document}
